\documentclass{article}

% KOMMANDOER
\def \t {\cdot}
\def \np {\newpage}
\def \q {\quad}
\def \ra {\rightarrow}
\def \imps {\Rightarrow}
\def \iffqq {\iff \qquad}
\def \impsqq {\implies \qquad}
\def \impdqq {\impliedby \qquad}
\def \kl {\text{kl}}
\def \d {\partial}
\def \X {\mathcal{X}}
\def \dvc {d_\text{VC}}
\def \dvch {d_\text{VC}(\mathcal{H})}

\newcommand\hash{\scalebox{1.2}{\raisebox{0ex}{\#}}}
\newcommand{\B} {\mathcal{B}}
\newcommand{\N}{\mathbb{N}}
\newcommand{\Q}{\mathbb{Q}}
\newcommand{\R}{\mathbb{R}}
\newcommand{\Z}{\mathbb{Z}}
\newcommand{\Oh}{\mathcal{O}}
\renewcommand{\H}{\mathcal{H}}
\newcommand{\Y}{\mathcal{Y}}
\newcommand{\E}[1]{\mathbb{E}\brks{#1}}
\newcommand{\Esub}[2]{\mathbb{E}_#1\brks{#2}}
\renewcommand{\P}[1]{\mathbb{P}\parr{#1}}
\newcommand{\eps}{\varepsilon}
\newcommand{\KL}[2]{\text{KL}\parr{#1 || #2}}
%\newcommand{\C}{\mathbb{C}}

\newcommand{\set}[1]{\left \{#1\right\}}
\newcommand{\abs}[1]{\left\vert #1 \right\vert}
\newcommand{\norm}[1]{\lVert #1 \rVert}
\newcommand{\parr}[1]{\left( #1 \right)}
\newcommand{\brks}[1]{\left[ #1 \right]}

\newcommand{\dotp}[1]{\langle #1 \rangle}
\newcommand{\amin}[1]{\underset{#1}{\text{argmin}}}
\newcommand{\amax}[1]{\underset{#1}{\text{argmax}}}
\newcommand{\m}[1]{\begin{pmatrix}#1\end{pmatrix}}
\newcommand{\ita}[1]{\textit{#1}}
\newcommand{\ms}[1]{\texttt{#1}}
\newcommand{\fat}[1]{\text{\boldmath$#1$}}
\newcommand{\qed}{\hfill\square}
\newcommand{\diff}[1]{\frac{d}{d{#1}}}
\newcommand{\pdiff}[1]{\frac{\d}{\d{#1}}}
\newcommand{\pdiffpow}[2]{\frac{\d^{#2}}{\d{#1}^{#2}}}

% independence (indp is the command)
\newcommand\indp{\protect\mathpalette{\protect\independenT}{\perp}}
\def\independen#1#2{\mathrel{\rlap{$#1#2$}\mkern2mu{#1#2}}}

\providecommand{\ceil}[1]{\left \lceil #1 \right \rceil }
\providecommand{\floor}[1]{\left \lfloor #1 \right \rfloor }

\newcommand{\minover}[1]{\underset{#1}{\min}}
\newcommand{\maxover}[1]{\underset{#1}{\max}}
\newcommand{\supover}[1]{\underset{#1}{\sup}}

% Load the listings package
\usepackage{listings}

% Definition of a style for code, matter of taste
\lstdefinestyle{mystyle}{
  language=Python,
  basicstyle=\ttfamily\footnotesize,
  % backgroundcolor=\color[HTML]{F7F7F7},
  % rulecolor=\color[HTML]{EEEEEE},
  % identifierstyle=\color[HTML]{24292E},
  % emphstyle=\color[HTML]{005CC5},
  % keywordstyle=\color[HTML]{D73A49},
  % commentstyle=\color[HTML]{6A737D},
  % stringstyle=\color[HTML]{032F62},
  emph={@property,self,range,True,False},
  morekeywords={super,with,as,lambda},
  % literate=%
  %   {+}{{{\color[HTML]{D73A49}+}}}1
  %   {-}{{{\color[HTML]{D73A49}-}}}1
  %   {*}{{{\color[HTML]{D73A49}*}}}1
  %   {/}{{{\color[HTML]{D73A49}/}}}1
  %   {=}{{{\color[HTML]{D73A49}=}}}1
  %   {/=}{{{\color[HTML]{D73A49}=}}}1,
  breakatwhitespace=false,
  breaklines=true,
  captionpos=b,
  keepspaces=true,
  numbers=none,
  showspaces=false,
  showstringspaces=false,
  showtabs=false,
  tabsize=4,
  frame=single,
  xleftmargin=\parindent,
  framexleftmargin=\parindent
}

% Apply the style globally
\lstset{style=mystyle}

\sloppy
\makeatletter
\makeatother

%for ex. cross ref. for equations
\usepackage{amssymb}
\usepackage{amsmath}
\usepackage{graphicx}
% \usepackage{amsthm}
% \usepackage{textcomp}
% \usepackage[utf8]{inputenc}
% \usepackage{blkarray}
% \usepackage{subcaption}
% \usepackage{gensymb}
% \usepackage[T1]{fontenc}
% \usepackage[margin=2.5cm]{geometry}
\usepackage{listings}
\usepackage{xcolor}
% \usepackage{xcolor}
\usepackage{nicefrac}
% \usepackage{pifont}
% \usepackage{amsmath,mathtools}
% \usepackage{amsfonts}
% \usepackage{stmaryrd}
% \usepackage{hyperref}
% \hypersetup{
%   colorlinks=true,
%   linkcolor=black,
%   filecolor=magenta,
%   urlcolor=cyan,
%   }

% Citations
% \usepackage{biblatex}
% \renewbibmacro{in:}{, version:}
% \addbibresource{citations.bib} %Import the bibliography file


% Add margin adjustments
\usepackage[margin=1in]{geometry}  % Reduced margins
\usepackage{titling}

% Make title inline and right-aligned
\pretitle{\begin{flushright}\large}
\posttitle{: }  % Just a colon and space after title
\preauthor{}    % No special formatting for author
\postauthor{, } % Comma and space after author
\predate{}      % No special formatting for date
\postdate{\end{flushright}}  % End the right alignment after date
\setlength{\droptitle}{-4em}  % Reduce space before title

% Reduce title spacing
\pretitle{\begin{center}\large}
\posttitle{\end{center}\vspace{-1em}}  % Reduce space after title
\preauthor{\begin{center}\normalsize}
\postauthor{\end{center}\vspace{-3em}}  % Reduce space after author
\predate{\begin{center}\normalsize}
\postdate{\end{center}\vspace{-0.5em}}  % Reduce space after date

% Add section formatting adjustments
\usepackage{titlesec}
\titleformat*{\section}{\large\bfseries}  % Make section headers large and bold
\titlespacing{\section}{0pt}{1em}{0.5em}  % Reduce spacing before and after section headers

\begin{document}
\title{DM887 Assignment 2: Game AI with Deep Expected SARSA}
\author{Victor Kaplan Kjellerup (vikje24)}
\date{}  % Add this line to prevent date from being printed
\maketitle

\definecolor{dgreen}{rgb}{0,0.75,0}

My algorithm design is a modification to the repository code found at https://github.com/iewug/Atari-DQN.

\section{Q-network design}
I used the regular (non-duel) DQN from the repository as a starting point, but made it smaller to save some amount of computational resources. All models compared in the following sections uses the same DQN. The input to the model is a stacked tensor consisting of four observations \(\text{obs}_t, \text{obs}_{t-1}, \text{obs}_{t-2}, \text{obs}_{t-3}\), each being a monochrome frame extended to quadratic size \(84 \times 84\) \(\text{pixels}^2\). The use of four observations provides the network with temporal information necessary to indicate velocity and direction of objects, for instance.

The network consists of three 2-d convolutional and two fully connected linear layers. All but the last layer uses Relu activation. The convolutional layers have out-channels [16, 24, 32], kernel sizes [8, 4, 3] and strides [4, 2, 1], respectively. This leads to smaller-resolution, but richer feature representations for each layer. The final layer has 32 ``latent frames'' of size \(7 \times 7\). The fully connected linear layers have out-channels \([128, |\mathcal{A}|]\), \(\mathcal{A}\) being the action space for the game being played. Their role is to distill the rich encoding into first 128 latent features, and finally one estimated value for each action.

\section{Deep Expected SARSA with Weighted Importance Sampling}
I use a replay buffer into which the realized \((s, a, r, s')\) events are pushed and from which random batches are sampled during model optimization. Due to occasional memory issues, I used a relatively small buffer size of 25,000 steps.

\begin{algorithm}
    \caption{Learning algorithm. Text specific to \textcolor{dgreen}{Q-learning in green}. Text specific to \textcolor{purple}{Deep Expected SARSA in purple} and \textcolor{blue}{blue when weighted importance sampling is used}.}
    \begin{algorithmic}[1]
        \Require $s_0$ (initial state), $\gamma$ (discount factor)
        \Ensure Updated $Q$ values
        \State Warm-up: collect 1000 examples of $(s, a, p(a|s), r, s') \rightarrow \mathcal{D}$, using $p(a|s) = \frac{1}{|A|} \quad \forall a, s$
        \State Initialize $Q^\pi$, $Q^-$ with weights \(\theta\) randomly
        \State \(\mathcal{D} \gets \emptyset\)
        \Repeat
            \State $s \gets \text{env.reset()}$
            \Repeat
                \State $\pi(\cdot | s) \gets \Call{eps-greedy}{Q_\theta, \eps}$
                \State $a \sim \pi(\cdot|s)$
                \State $p(a|s) \gets (1 - \eps) \cdot \mathbbm{1}_{\set{a\text{ is the greedy choice}}} + \frac{\eps}{|A|}$
                \State $r, s' \gets \text{env.step}(s, a)$
                \State $\mathcal{D} \cup (s, a, p(a|s), r, s'),\qquad \tilde{\mathcal{D}} \overset{\ita{iid}}{\sim}\mathcal{D}$
                \State $s \gets s'$
                \State $Q^\pi \gets \Call{update-Q}{\tilde{\mathcal{D}}, Q^\pi, Q^-}$
                \If{1000 steps have been played since last \(Q^-\)-update}
                    \State $Q^- \gets Q^\pi$
                \EndIf
            \Until{episode end}
        \Until{end of simulation (fixed epoch)}
        
        \Procedure{update-Q}{$\mathcal{D}, Q^\pi, Q^-$}
            \State $L \gets \emptyset$
            \For{$(s, a, p(a|s), r, s') \in \mathcal{D}$}
                \State $\textcolor{dgreen}{\hat{y} \gets r + \gamma \arg\max_{a'}\set{Q^-(s',a')}}$
                \State $\textcolor{purple}{\hat{y} \gets r + \gamma \sum\limits_{a' \in \mathcal{A}} \pi(a' | s') Q^-(s', a')}$
                \State $L \cup \textcolor{blue}{\frac{\pi(a|s)}{p(a|s)}}(Q^\pi(s, a) - \hat{y})^2, \qquad \textcolor{blue}{(\pi(a|s) = (1 - \eps) \cdot \mathbbm{1}_{\set{a\text{ is the greedy choice}}} + \frac{\eps}{|A|})}$
                \EndFor
            \State Update $Q^\pi$ parameters \(\theta\) using Adaptive Momentum Gradient Estimation on \(L\)
            \State \Return $Q^\pi$
        \EndProcedure
    \end{algorithmic}
\end{algorithm}

\section{Comparing methods}
\section{The effects of weighted importance sampling}
\section{Limitations and future considerations}

\end{document}
